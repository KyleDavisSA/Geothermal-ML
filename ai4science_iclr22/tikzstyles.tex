% library

%\colorlet{incolor}{tumorange}
%\colorlet{outcolor}{tumblue}

\usepackage{tikz-dimline}

\tikzset{inNode/.style={draw=incolor, fill=incolor, circle, inner sep=1.5pt}}
\tikzset{outNode/.style={draw=outcolor,  fill=outcolor, circle, inner sep=1pt}}
\tikzset{projNode/.style={draw=outcolor, fill=outcolor, circle, inner sep=1pt}}

\usetikzlibrary{shapes.arrows, fadings, calc, positioning}
\tikzfading[name=fade right,
  left color=transparent!0, right color=transparent!100]
\tikzfading[name=fade left,
  left color=transparent!100, right color=transparent!0]
\tikzfading[name=fade top,
  bottom color=transparent!0, top color=transparent!100]
  
\pgfdeclarelayer{bg}    % declare background layer
\pgfdeclarelayer{fg}    % declare foreground layer
\pgfsetlayers{bg,main,fg}  % set the order of the layers (main is the standard layer)

\definecolor{participantAcolor}{RGB}{243,98,33}
\definecolor{participantBcolor}{RGB}{0,102,189}
%\definecolor{participantCcolor}{RGB}{253,215,130}
\definecolor{participantCcolor}{RGB}{66,70,50}
\definecolor{cplColor}{RGB}{0,0,0}
\definecolor{preciceColor}{RGB}{ 161, 177, 25}

%%% CI colors, https://portal.mytum.de/corporatedesign/regeln/index_styleguide p.20

\definecolor{ci_pantone300}{RGB}{ 0, 101, 189}

\definecolor{ci_pantone301}{RGB}{ 0, 82, 147}
\definecolor{ci_pantone540}{RGB}{ 0, 51, 89}

\definecolor{ci_pantone283}{RGB}{ 152, 198, 234}
\definecolor{ci_pantone542}{RGB}{ 100, 160, 200}
\definecolor{ci_ivory}{RGB}{ 218, 215, 203}
\definecolor{ci_orange}{RGB}{ 227, 114, 34}
\definecolor{ci_green}{RGB}{ 162, 173, 0}  
